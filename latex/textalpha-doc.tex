\documentclass[a4paper, parskip=true]{scrartcl}
\usepackage{booktabs}

\ifdefined \UnicodeEncodingName % set by LaTeX for Unicode-aware engines
  % Setup for Unicode fonts (Xe-/LuaTeX)
  \usepackage{fontspec}
  \setmainfont{Linux Libertine O}
  \setsansfont{Linux Biolinum O}
  \newcommand*{\greekfontencoding}{TU}
\else
  % Setup for 8-bit fonts (pdfTeX/LuaTeX)
  % (XeTeX in compatibility mode would require inputenc hacks and is not
  % reliable.)
  \usepackage{lmodern}
  \usepackage[LGR,T1]{fontenc}
  \newcommand*{\greekfontencoding}{LGR}
  \newcommand*{\latinencoding}{T1}
\fi

\usepackage[pdfencoding=auto,colorlinks=true,linkcolor=blue]{hyperref}
\usepackage{bookmark}
\makeatletter
\providecommand*{\href}{\@secondoftwo}
\providecommand*{\url}{\texttt}
\makeatother

\usepackage[normalize-symbols, % comment option out to test error reporting
            keep-semicolon%
           ]{textalpha}

% auxiliary definitions:
\ProvideTextCommandDefault{\textvarstigma}{}
\newcommand{\cs}[1]{\texttt{\textbackslash#1}}


\begin{document}

\title{The \emph{textalpha} package}
\author{Günter Milde}
\date{2020/10/30}

\maketitle

\abstract{\noindent
The \emph{textalpha} package enables the use of Greek characters
in text independent of font encoding or TeX engine.%
\footnote{
  This document was compiled using
  \ifdefined \UTFencname % defined by fontspec
     Unicode fonts (font encoding \latinencoding).
     For a version using 8-bit fonts, see
     \href{textalpha-doc.pdf}{textalpha-doc.pdf}.
  \else
     8-bit fonts (font encoding \latinencoding).
     For a version using Unicode fonts, see
     \href{textalpha-tu.pdf}{textalpha-tu.pdf}.
  \fi
}
Input is possible via text commands (\cs{textalpha} \ldots \cs{textOmega})
or Unicode literals\footnote{\label{requires-greek-inputenc}
  Requires \emph{\href{https://ctan.org/pkg/greek-inputenc}{greek-inputenc}}
  or XeTeX/LuaTeX.}.
} % end abstract

\tableofcontents


\section{Usage}

Load this package in the preamble of your document with
\begin{verbatim}
  \usepackage[<options>]{textalpha}
\end{verbatim}
Now you are ready to use literal Unicode
characters\footref{requires-greek-inputenc} or the \cs{textalpha} \ldots
\cs{textOmega} macros anywhere in the text.\footnote{
  Using the shorter \cs{alpha} \ldots \cs{Omega} macros (known from math mode)
  is possible with the \emph{\href{alphabeta-doc.pdf}{alphabeta}} package.}
See the source of this document \texttt{textalpha-doc.tex} for a setup and
usage example and \href{greek-fontenc-doc.html}{greek-fontenc-doc} for
links to additional documentation.


\subsection{Options}
\subsubsection{\texttt{normalize-symbols}}

Mathematical notation uses variant shapes of some Greek letters as
additional symbols.  There are separate code points for the symbol variants
in Unicode. TeX supports some of the variant shape symbols in mathematical
mode ($\theta|\vartheta, \phi|\varphi, \pi|\varpi, \rho|\varrho,
\epsilon|\varepsilon$) but not in the LGR font encoding used for Greek text
in 8-bit TeX.
The variations have no syntactic meaning in Greek text and text fonts may
use the variant shapes in place of the “regular” ones as a stylistic choice.

The \texttt{normalize-symbols} option merges letters and symbols to Greek
letters.
This way, text copied from external sources can be compiled without
errors even if it contains a GREEK SYMBOL … in place of a GREEK LETTER …
\begin{quote}
  The source of this paragraph uses both variants for beta (β|ϐ),
  theta (θ|ϑ), phi (φ|ϕ), pi (π|ϖ), kappa (κ|ϰ), rho (ρ|ϱ), Theta (Θ|ϴ),
  and epsilon (ε|ϵ).
\end{quote}
%
This option is ignored with Unicode fonts.

\begin{description}
\item [Attention:] Do not use this option in cases where the distinction
  between the symbol variants may be important (e.g. in a mathematical or
  scientific context). Use the respective characters in mathematical mode
  or XeTeX/LuaTeX with Unicode fonts.
\end{description}

\subsubsection{\texttt{keep-semicolon}}

LGR is no \href{https://mirrors.ctan.org/macros/latex/base/encguide.pdf}%
{standard text font encoding}.
Latin characters and some other ASCII symbols are mapped to Greek
``equivalents'' if LGR is the active font encoding. (See
\href{https://mirrors.ctan.org/language/babel/contrib/greek/babel-greek-doc.html#lgr-latin-transliteration}%
{babel-greek} for a description of this Latin-Greek transliteration.)

Special care is required with the question mark characters: The LGR font
encoding uses the Latin question mark as input for the \emph{erotimatiko}
and maps the semicolon to a middle dot (\emph{ano teleia}).
As a result, Unicode-encoded texts that use the semicolon as
\emph{erotimatiko} end up with an \emph{ano teleia} in its place!
Without special care, only the deprecated character 037E GREEK QUESTION MARK%
\footnote{The Unicode standard provides the code point 037E GREEK QUESTION MARK
        but says character 003B SEMICOLON and not 037E is the preferred
        character for a `Greek question mark' (erotimatiko).}
works with both, Xe/LuaTeX and 8-bit TeX.

The \verb|\textsemicolon| command inserts an \emph{erotimatiko} in LGR and a
semicolon else (i.e. always a character that looks like a semicolon):
\begin{quote}
  Latin (\latinencoding) a\textsemicolon{} b,
  Greek (\greekfontencoding) \ensuregreek{a\textsemicolon{} b}
\end{quote}

With the \texttt{keep-semicolon} option, character 003B SEMICOLON can be used
for the \emph{erotimatiko} also with LGR encoded fonts:

\begin{center}
\begin{tabular}{ccl}
  Latin (\latinencoding) & Greek (\greekfontencoding) & question mark character \\
  \midrule
  ; & \ensuregreek{;} & 037E GREEK QUESTION MARK \\
  ; & \ensuregreek{;} & 003B SEMICOLON \\
  ? & \ensuregreek{?} & 003F QUESTION MARK \\
\end{tabular}
\end{center}

This option is ignored with Unicode fonts (where the SEMICOLON literal
always prints a semicolon character).

Test whether this works as expected in math mode:
\ensuregreek{$a b; a\;b, (\mathrm{a;}\textrm{a;}2)$}.


\subsection{Symbol macros for Breathings}

\emph{textalpha} defines the macros \cs{<} and \cs{>} for the
\href{https://en.wikipedia.org/wiki/Rough_breathing}{dasia} (rough breathing)
and \href{https://en.wikipedia.org/wiki/Smooth_breathing}{psili} (smooth
breathing) diacritics.


\section{Limitations \label{sec:limitations}}

If Greek letters are used while the active font encoding does not support
Greek, the internal font encoding switches interfere with other work behind
the scenes.
Kerning, diacritics and up/down-casing show problems that can be avoided by
\begin{itemize}
  \item use of \emph{babel} and the correct language setting,
  \item an explicit font encoding switch,
        e.g., wrapping in \cs{ensuregreek}\footnote{
          The \cs{ensuregreek} macro ensures the argument
          is set in a font encoding supporting Greek
          without adverse side-effects if the active font encoding is
          already LGR or TU.}, or
  \item XeTeX/LuaTeX with Unicode fonts.
\end{itemize}
%
\ifdefined\UnicodeEncodingName
  For details, see \href{textalpha-doc.pdf}{textalpha-doc.pdf}.
\else

  \subsection{Kerning}

  With pdfTeX and 8-bit fonts, no kerning occurs between Greek characters in
  non-Greek text due to the internal font encoding switch:
  \begin{quote}
    \textAlpha\textUpsilon\textAlpha{} (\latinencoding) vs.
    \ensuregreek{\textAlpha\textUpsilon\textAlpha} (\greekfontencoding).
  \end{quote}
  Compiling with LuaTeX provides kerning also on font encoding boundaries.


  \subsection{Diacritics}

  With 8-bit TeX, accent macros do not work with Unicode literals as base
  character. Use the Latin transliteration or LICR commands.

  \medskip\noindent
  Composition of diacritics (like \verb|\accdasia\acctonos| or \cs{<\'})
  fails in other font encodings. Long names (like \cs{accdasiaoxia}) work.
  \begin{quote}
    \<'\textalpha{} vs.
    \ensuregreek{\<'\textalpha} (\greekfontencoding)
  \end{quote}
  %
  With LGR and TU, pre-composed glyphs are chosen if available. In other font
  encodings, accent macros do not select pre-composed characters.
  The difference is a sub-optimal placement of the accent and becomes
  obvious if you drag-and-drop text from the PDF version of this document.:
  \begin{quote}
    \accdasiaoxia\textalpha{} (\latinencoding) vs.
    \ensuregreek{\accdasiaoxia\textalpha{}} (\greekfontencoding).
  \end{quote}
  %
  In Greek typographical practice, diacritics (except the dialytika and
  sub-iota) are placed before capital letters in Titlecase (Ἀρχιμήδης) and
  dropped in uppercase (ΑΡΧΙΜΗΔΗΣ).

  Diacritics input via standard accent macros are misplaced
  if the active font encoding does not support Greek.

  With the \cs{MakeUppercase} implementation introduced 2022/06, Greek
  upcasing rules are only applied to literal characters if the text language
  is set to Greek with Babel and to standard accent macros if the documents
  loads Greek with Babel (i.e. not in this document).\footnote{
    With the pre-2022 \cs{MakeUppercase} implementation, the above rules
    were fully applied if the active font encoding is LGR or TU.}
  \begin{quote}
    \begin{tabular}{cccc}
      & named accent & standard accent & literal \\ \midrule
    \greekfontencoding
      & \ensuregreek{\acctonos\textAlpha{} → \MakeUppercase{\acctonos\textAlpha}}
      & \ensuregreek{\'\textAlpha{}        → \MakeUppercase{\'\textAlpha}}
      & \ensuregreek{Ά                     → \MakeUppercase{Ά}} \\
    \latinencoding
      &              \acctonos\textAlpha{} → \MakeUppercase{\acctonos\textAlpha}
      &              \'\textAlpha{}        → \MakeUppercase{\'\textAlpha}
      &              Ά                     → \MakeUppercase{Ά}
    \end{tabular}
  \end{quote}

  The dialytika marks a \emph{hiatus} (break-up of a diphthong). It must be
  present in UPPERCASE even where it is redundant in lowercase (the hiatus can
  also be marked by an accent or breathing on the first of two consecutive
  vowels). The auto-hiatus feature works in LGR and TU font encodings only:
  \begin{quote}
    \newcommand*{\sample}{%
      \acctonos\textalpha\textupsilon{}, \acctonos\textepsilon\textiota{}}
    \sample{} → \MakeUppercase{\sample} (\latinencoding) vs.
    \ensuregreek{\sample{} → \MakeUppercase{\sample}} (\greekfontencoding)
  \end{quote}

  With the old implementation of \cs{MakeUppercase}, the auto-hiatus
  feature works with LICR macros but not Unicode literals.
  The new implementation works with Unicode literals, too, but only if the text
  language is Greek (i.e. not in this document).
  \begin{quote}
  \ensuregreek{%
  \accpsili\textalpha\textupsilon\textpi\textnu\acctonos\textiota\textalpha}
  $\mapsto$ \ensuregreek{\MakeUppercase{%
  \accpsili\textalpha\textupsilon\textpi\textnu\acctonos\textiota\textalpha}}
  (LICR macros: OK with LGR or TU)

  \ensuregreek{ἀυπνία} $\mapsto$
  \ensuregreek{\MakeUppercase{ἀυπνία}} (literal characters: fails without Babel)
  \end{quote}
\fi

\section{Test and Examples}

\subsection{Greek alphabet}

Greek literal characters in Latin text (font encoding \latinencoding):
\begin{quote}
  α β γ δ ε ζ η θ ι κ λ μ ν ξ ο π ρ σ ς τ υ φ χ ψ ω

  Α Β Γ Δ Ε Ζ Η Θ Ι Κ Λ Μ Ν Ξ Ο Π Ρ Σ   Τ Υ Φ Χ Ψ Ω
\end{quote}
%
Greek letters via default macros in Latin text (font encoding \latinencoding):
%
\newcommand*{\greekAlphabetsample}{
  \textAlpha{} \textBeta{} \textGamma{} \textDelta{} \textEpsilon{}
  \textZeta{} \textEta{} \textTheta{} \textIota{} \textKappa{}
  \textLambda{} \textMu{} \textNu{} \textXi{} \textOmicron{} \textPi{}
  \textRho{} \textSigma{} \textTau{} \textUpsilon{} \textPhi{}
  \textChi{} \textPsi{} \textOmega{}
  }
\newcommand*{\greekalphabetsample}{
  \textalpha{} \textbeta{} \textgamma{} \textdelta{} \textepsilon{}
  \textzeta{} \texteta{} \texttheta{} \textiota{} \textkappa{}
  \textlambda{} \textmu{} \textnu{} \textxi{} \textomicron{} \textpi{}
  \textrho{} \textsigma{} \textvarsigma{} \texttau{} \textupsilon{}
  \textphi{} \textchi{} \textpsi{} \textomega{}
}
\begin{quote}
  \greekalphabetsample

  \greekAlphabetsample
\end{quote}
%
\ifdefined\UnicodeEncodingName
\else
  Greek letters via Latin transliteration (works only in LGR font encoding):
  \begin{quote}
    \ensuregreek{a b g d e z h j i k l m n x o p r sv c t u f q y w}

    \ensuregreek{A B G D E Z H J I K L M N X O P R S T U F Q Y W}
  \end{quote}
\fi
%
Archaic Greek letters and Greek punctuation

\newcommand*{\archaicgreeksample}{
  \textdigamma \textDigamma{}
  \textkoppa   \textKoppa{}
  \textqoppa   \textQoppa{}
  \textsampi   \textSampi{}
  \textstigma
  \textvarstigma % only in LGR
  \textStigma{}
  \textanoteleia{}
  \texterotimatiko{}
  \textdexiakeraia{}
  \textaristerikeraia{}
}
\begin{quote}
  \archaicgreeksample
\end{quote}
%
Diacritics

\begin{quote}
  Short macros:\footnote{
    Composite diacritics require wrapping in \cs{ensuregreek}.}
  \"{} \'{} \`{} \~{} \<{} \>{} \u{} \={}
  \ensuregreek{\"~{} \"'{} \"`{} \<~{} \<`{} \<'{} \>~{} \>'{} \>`{}}

  Named macros:
  \accdialytika{}
  \acctonos{}
  \accvaria{}
  \accperispomeni{}
  \accdasia{}
  \accpsili{}
  \ypogegrammeni{}
  \prosgegrammeni{}
  %
  \accdialytikaperispomeni{}
  \accdialytikatonos{}
  \accdialytikavaria{}
  \accdasiaperispomeni{}
  \accdasiavaria{}
  \accdasiaoxia{}
  \accpsiliperispomeni{}
  \accpsilioxia{}
  \accpsilivaria{}
  \ifdefined\UnicodeEncodingName
  \else

    Only in LGR:
    \accinvertedbrevebelow{} % == \textsubarch{}
    \accbrevebelow{}
  \fi
\end{quote}

\medskip\noindent
Accent macros can start with ``\verb|\a|'' instead of ``\verb|\|'' when the
short form is redefined, e.\,g. inside a \emph{tabbing} environment.
This also works for the new-defined Dasia and Psili shortcuts:
\begin{quote}
  \begin{tabbing}
       col 1\quad \= col 2\quad \= col 3\quad \= col 4\quad \\
       Viele      \> Gr\a"u\ss e
       \> \greekscript \a<\textalpha{}
       \> \greekscript \a>\textomega
  \end{tabbing}
\end{quote}

\subsubsection{Sigma}

The lower Sigma comes in two variants: \verb|\textsigma| \textsigma{} is
used inside a word and \verb|\textfinalsigma| \textfinalsigma{} (or
\verb|\textvarsigma| \textvarsigma{}) at the end of words.

In LGR, the Latin letter \verb|s| and the command \verb|\textautosigma|
print the ``normal'' sigma if followed by another letter and the final sigma
if followed by space or punctuation. This is implemented via the font
ligature mechanism in LGR\footnote{
  TODO: Fix \cs{textautosigma} with Unicode fonts.}:
\begin{quote}
  \ensuregreek{\textautosigma\textautosigma} (\greekfontencoding) vs.
  \textautosigma{}\textautosigma{} (\latinencoding).
\end{quote}
The upper case of both sigma variants is \verb|\textSigma|, the lower case
of \cs{textSigma} is \cs{textautosigma}.

\medskip\noindent
\begin{samepage}
Test Unicode literal and \verb|\text...| commands:
\begin{quote}
  \newcommand{\sample}{σ\textsigma{}
                       ς\textvarsigma \textfinalsigma \textautosigma{}
                       ΣΣ \textSigma\textSigma{}}
  \begin{tabular}{ll}
    no change:                          & \sample \\
    MakeUppercase: 		      & \MakeUppercase{\sample} \\
    MakeLowercase (\latinencoding):     & \MakeLowercase{\sample} \\
    MakeLowercase (\greekfontencoding): & \ensuregreek{\MakeLowercase{\sample}}
  \end{tabular}
\end{quote}
\end{samepage}

\subsection{Greek literal characters in non-Greek text}

With the \emph{textalpha} package,
\href{https://ctan.org/pkg/greek-inputenc}{greek-inputenc} and input
encoding \texttt{utf8}, Greek Unicode literals can be used in text with
any font encoding. See Tables \ref{tab:greek-and-coptic} and
\ref{tab:greek-extended}.

Kerning is preserved if the active font encoding supports Greek. This can be
secured by wrapping the Greek text part in \verb|\ensuregreek| or setting the
text language with Babel: \ensuregreek{AΫA}

\begin{table}[tbp]
\setlength{\tabcolsep}{0.45em}
\centerline{
\begin{tabular}{rrrrrrrrrrrrrrrrr}
\toprule
    & 0 & 1 & 2 & 3 & 4 & 5 & 6 & 7 & 8 & 9 & A & B & C & D & E & F\\
\midrule
370 & ◦ & ◦ & ◦ & ◦ & ʹ & ͵ & ◦ & ◦ &   &   & ͺ & ◦ & ◦ & ◦ & ; &  \\
380 &   &   &   &   & ΄ & ΅ & Ά & · & Έ & Ή & Ί &   & Ό &   & Ύ & Ώ\\
390 & ΐ & Α & Β & Γ & Δ & Ε & Ζ & Η & Θ & Ι & Κ & Λ & Μ & Ν & Ξ & Ο\\
3A0 & Π & Ρ &   & Σ & Τ & Υ & Φ & Χ & Ψ & Ω & Ϊ & Ϋ & ά & έ & ή & ί\\
3B0 & ΰ & α & β & γ & δ & ε & ζ & η & θ & ι & κ & λ & μ & ν & ξ & ο\\
3C0 & π & ρ & ς & σ & τ & υ & φ & χ & ψ & ω & ϊ & ϋ & ό & ύ & ώ &  \\
3D0 & ◦ & ◦ & ◦ & ◦ & ◦ & ◦ & ◦ & ◦ & Ϙ & ϙ & Ϛ & ϛ & Ϝ & ϝ & Ϟ & ϟ\\
3E0 & Ϡ & ϡ & ◦ & ◦ & ◦ & ◦ & ◦ & ◦ & ◦ & ◦ & ◦ & ◦ & ◦ & ◦ & ◦ & ◦\\
3F0 & ◦ & ◦ & ◦ & ◦ & ◦ & ◦ & ◦ & ◦ & ◦ & ◦ & ◦ & ◦ & ◦ & ◦ & ◦ & ◦\\
\bottomrule
\end{tabular}
} % end centerline
\caption{Greek and Coptic Unicode Block, input as literal Unicode
   characters in \latinencoding{} font encoding
   (legend: ◦ glyph missing in LGR).}
\label{tab:greek-and-coptic}
\end{table}


\begin{table}[tbp]
\setlength{\tabcolsep}{0.45em}
\centerline{
\begin{tabular}{rrrrrrrrrrrrrrrrr}
\toprule
     & 0 & 1 & 2 & 3 & 4 & 5 & 6 & 7 & 8 & 9 & A & B & C & D & E & F\\
\midrule
1F00 & ἀ & ἁ & ἂ & ἃ & ἄ & ἅ & ἆ & ἇ & Ἀ & Ἁ & Ἂ & Ἃ & Ἄ & Ἅ & Ἆ & Ἇ\\
1F10 & ἐ & ἑ & ἒ & ἓ & ἔ & ἕ &   &   & Ἐ & Ἑ & Ἒ & Ἓ & Ἔ & Ἕ &   &  \\
1F20 & ἠ & ἡ & ἢ & ἣ & ἤ & ἥ & ἦ & ἧ & Ἠ & Ἡ & Ἢ & Ἣ & Ἤ & Ἥ & Ἦ & Ἧ\\
1F30 & ἰ & ἱ & ἲ & ἳ & ἴ & ἵ & ἶ & ἷ & Ἰ & Ἱ & Ἲ & Ἳ & Ἴ & Ἵ & Ἶ & Ἷ\\
1F40 & ὀ & ὁ & ὂ & ὃ & ὄ & ὅ &   &   & Ὀ & Ὁ & Ὂ & Ὃ & Ὄ & Ὅ &   &  \\
1F50 & ὐ & ὑ & ὒ & ὓ & ὔ & ὕ & ὖ & ὗ &   & Ὑ &   & Ὓ &   & Ὕ &   & Ὗ\\
1F60 & ὠ & ὡ & ὢ & ὣ & ὤ & ὥ & ὦ & ὧ & Ὠ & Ὡ & Ὢ & Ὣ & Ὤ & Ὥ & Ὦ & Ὧ\\
1F70 & ὰ & ά & ὲ & έ & ὴ & ή & ὶ & ί & ὸ & ό & ὺ & ύ & ὼ & ώ &   &  \\
1F80 & ᾀ & ᾁ & ᾂ & ᾃ & ᾄ & ᾅ & ᾆ & ᾇ & ᾈ & ᾉ & ᾊ & ᾋ & ᾌ & ᾍ & ᾎ & ᾏ\\
1F90 & ᾐ & ᾑ & ᾒ & ᾓ & ᾔ & ᾕ & ᾖ & ᾗ & ᾘ & ᾙ & ᾚ & ᾛ & ᾜ & ᾝ & ᾞ & ᾟ\\
1FA0 & ᾠ & ᾡ & ᾢ & ᾣ & ᾤ & ᾥ & ᾦ & ᾧ & ᾨ & ᾩ & ᾪ & ᾫ & ᾬ & ᾭ & ᾮ & ᾯ\\
1FB0 & ᾰ & ᾱ & ᾲ & ᾳ & ᾴ &   & ᾶ & ᾷ & Ᾰ & Ᾱ & Ὰ & Ά & ᾼ & ᾽ & ι & ᾿\\
1FC0 & ῀ & ῁ & ῂ & ῃ & ῄ &   & ῆ & ῇ & Ὲ & Έ & Ὴ & Ή & ῌ & ῍ & ῎ & ῏\\
1FD0 & ῐ & ῑ & ῒ & ΐ &   &   & ῖ & ῗ & Ῐ & Ῑ & Ὶ & Ί &   & ῝ & ῞ & ῟\\
1FE0 & ῠ & ῡ & ῢ & ΰ & ῤ & ῥ & ῦ & ῧ & Ῠ & Ῡ & Ὺ & Ύ & Ῥ & ῭ & ΅ & `\\
1FF0 &   &   & ῲ & ῳ & ῴ &   & ῶ & ῷ & Ὸ & Ό & Ὼ & Ώ & ῼ & ´ & ῾ &  \\
\bottomrule
\end{tabular}
} % end centerline
\caption{Greek Extended Unicode Block, input as literal Unicode
characters in \latinencoding{} font encoding.}
\label{tab:greek-extended}
\end{table}

Combined Diacritics work for pre-composed characters: ᾅ.

Diacritics (except diaeresis) are dropped with
MakeUppercase with LaTeX versions older than 2022/06
For other versions, set the language of to-be-upcased Greek text with Babel:
μαΐστρος, δύο  $\mapsto$ \MakeUppercase{μαΐστρος, δύο}.


\subsection{PDF strings}

With \emph{textalpha} and
\emph{\href{https://ctan.org/pkg/greek-inputenc}{greek-inputenc}}, there
are two options to get Greek letters in PDF strings: LICR macros and literal
Unicode input.

\subsubsection{\textlambda\textomicron\textgamma\textomicron\textvarsigma{},
               λογος, and \ensuregreek{logos}}

The subsection title above uses: LICR macros, Unicode input and the LGR
transliteration for the Greek word \ensuregreek{logos}.

LICR macros and Unicode literals work fine everywhere, the
Latin transliteration remains Latin in the PDF metadata
(sidebar table of contents in the PDF viewer) and with Xe/LuaTeX.

\subsubsection{\greekalphabetsample}

\subsubsection{\greekAlphabetsample}

\subsubsection{\archaicgreeksample}

\ifdefined \UnicodeEncodingName
  Archaic characters are missing in many fonts, including the ``Biolinum'' font
  used in this document.
\fi.
\end{document}
