
% ---
% Pacotes básicos
% ---
\usepackage{bookmark}				% Usa a fonte Bookman Old Style
\usepackage[T1]{fontenc}			% Selecao de codigos de fonte.
\usepackage[utf8]{inputenc}		% Codificacao do documento (conversão automática dos acentos)
\usepackage{color}				% Controle das cores
\usepackage{graphicx}			% Inclusão de gráficos
\usepackage{microtype} 			% para melhorias de justificação

\usepackage[brazilian,hyperpageref]{backref}	 % Paginas com as citações na bibl
\usepackage[alf]{abntex2cite}	% Citações padrão ABNT
\usepackage{float}
\usepackage{url}
\usepackage{enumerate}
\usepackage{siunitx}
\usepackage{setspace}
\usepackage[euler]{textgreek}
\usepackage{tikz}
\usetikzlibrary{arrows.meta, positioning, shapes.misc, backgrounds}

%%%%%%%%% NOVO
\usepackage{hyperref}
\hypersetup{
    colorlinks=true,
    linkcolor=blue,
    citecolor=blue,
    filecolor=magenta,      
    urlcolor=blue,
    bookmarksopen=true,
}
\urlstyle{same}

%%%   letra capitular
\usepackage{lettrine}
\usepackage{palatino}

%%%    tabelas - configuração
\usepackage{booktabs}
\usepackage{siunitx}

\usepackage{lscape}


%%% lista de abreviações e siglas automática 
\usepackage{acro}
\ExplSyntaxOn
\keys_define:nn { acro }
  {
    list/include-classes .code:n = \acsetup{list/include = {#1}},
    list/exclude-classes .code:n = \acsetup{list/exclude = {#1}},
    print-acronyms/include-classes .code:n = \acsetup{list/include = {#1}},
    print-acronyms/exclude-classes .code:n = \acsetup{list/exclude = {#1}}
  }
\ExplSyntaxOff

%%%  LISTA DE ABREVIAÇÕES E SÍMBOLOS

\acsetup{first-style=short}

% Abreviações
\DeclareAcronym{llm}{
	short = LLM,
	long  = \textit{Large Language Model},
	tag   = abbrev
}

\DeclareAcronym{ia}{
	short = IA,
	long  = Inteligência Artificial,
	tag   = abbrev
}

\DeclareAcronym{api}{
	short = API,
	long  = \textit{Application Programming Interface},
	tag   = abbrev
}

\DeclareAcronym{kpi}{
	short = KPI,
	long  = \textit{Key Performance Indicator},
	tag   = abbrev
}

\DeclareAcronym{mq}{
	short = MQ,
	long  = Fila de Mensageria (ex.: NATS ou Kafka),
	tag   = abbrev
}

% Símbolos
\DeclareAcronym{successrate}{
	short = $S_r$,
	long  = Taxa de sucesso agregada por rodada do benchmark,
	sort  = Sr,
	tag   = nomencl
}

\DeclareAcronym{costrun}{
	short = $C_{run}$,
	long  = Custo médio por execução completa,
	sort  = Crun,
	tag   = nomencl
}

\DeclareAcronym{latency}{
	short = $L_{p95}$,
	long  = Latência do percentil 95 por fluxo de avaliação,
	sort  = Lp95,
	tag   = nomencl
}

% Garante a impressão das listas mesmo sem citações explícitas
\acuse{llm}
\acuse{ia}
\acuse{api}
\acuse{kpi}
\acuse{mq}
\acuse{successrate}
\acuse{costrun}
\acuse{latency}



%%%%%% estilo do capítulo
%%%%%% escolha 1 e tire o comentário
%% acesse a página e escolha
%%http://ctan.math.washington.edu/tex-archive/info/latex-samples/MemoirChapStyles/MemoirChapStyles.pdf

%% primeiro
%\chapterstyle{madsen}

%% segundo
%\chapterstyle{southall}

%% terceiro
%\chapterstyle{Ger}

%% quarto
%\chapterstyle{verville}

%% quinto
\setlength\midchapskip{10pt}
\makechapterstyle{VZ23}{
    \renewcommand\chapternamenum{}
    \renewcommand\printchaptername{}
    \renewcommand\chapnumfont{\Huge\bfseries\centering}
    \renewcommand\chaptitlefont{\Huge\scshape\centering}
    \renewcommand\afterchapternum{%
        \par\nobreak\vskip\midchapskip\hrule\vskip\midchapskip}
    \renewcommand\printchapternonum{%
        \vphantom{\chapnumfont \thechapter}
        \par\nobreak\vskip\midchapskip\hrule\vskip\midchapskip}
}
\chapterstyle{VZ23}

%%%%%%%%%%%%%

\titulo{Avaliação de Agentes de IA em Benchmarks de 2025}
\autor{Matheus Serrão Uchoa}
\local{Manaus - AM}
\data{2025}
\orientador[Orientador(a)]{Prof. Dr. Nome do Orientador}%Atualizar com o nome do(a) orientador(a)
\instituicao{%
  Universidade Federal do Amazonas - UFAM
  \par
  Instituto de Computação- IComp}
\curso{%
  Programa Pós-Graduação em Informática - PPGI}
\tipotrabalho{Monografia}
% O preambulo deve conter o tipo do trabalho, o objetivo, 
% o nome da instituição e a área de concentração 

\preambulo{Monografia apresentada ao Instituto de Computação da Universidade Federal do Amazonas como requisito parcial para a obtenção do título de Bacharel em Ciência da Computação.}


% Informações do PDF
\makeatletter
\hypersetup{
    	%pagebackref=true,
	pdftitle={\@title}, 
	pdfauthor={\@author},
    	pdfsubject={\imprimirpreambulo},
    pdfcreator={LaTeX with abnTeX2},
	pdfkeywords={abnt}{latex}{abntex}{abntex2}{trabalho acadêmico},
	bookmarksdepth=4
}
\makeatother

% --- 
% Espaçamentos entre linhas e parágrafos 
% --- 

% O tamanho do parágrafo é dado por:
\setlength{\parindent}{1.3cm}

% Controle do espaçamento entre um parágrafo e outro:
%\setlength{\parskip}{0.2cm}  % tente também \onelineskip

%\setbeforesecskip{3em}
%\setbeforesubsecskip{3em}

% ---
% compila o indice
% ---
\makeindex
